\documentclass[12pt]{article}
\usepackage[utf8x]{inputenc}
\usepackage[frenchb]{babel}
\usepackage[T1]{fontenc}
\usepackage{eurosym}
%% Sets page size and margins
\usepackage[a4paper,top=3cm,bottom=2cm,left=2cm,right=2cm,marginparwidth=1.75cm]{geometry}
%% Useful packages
\usepackage{titlesec}

\usepackage[colorinlistoftodos]{todonotes}
\usepackage[hidelinks]{hyperref}
\usepackage{tocstyle}
\usepackage{PTSerif}
\usepackage{xcolor}
\usepackage{multirow}
\definecolor{bleu}{rgb}{0.08, 0.45, 0.53}

\titleformat{\section}
{\color{bleu}\normalfont\bfseries}
{\color{bleu}}{1em}{\MakeUppercase}

\renewcommand{\thesubsection}{\arabic{subsection}.}
\titleformat{\subsection}
{\color{bleu}\normalfont\bfseries}
{\hspace{3em}\color{bleu}\thesubsection}{1em}{}

\renewcommand{\thesubsubsection}{\alph{subsubsection}.}
\titleformat{\subsubsection}
{\color{bleu}\normalfont}
{\hspace{6em}\color{bleu}\thesubsubsection}{1em}{\MakeUppercase}

\usepackage{enumitem}
\setlist[itemize]{itemsep=6pt, label={$\bullet$}}

\usepackage{lastpage}
\usepackage{fancyhdr}
\pagestyle{fancy}
 \fancyhf{}
\renewcommand{\headrulewidth}{0pt}
\rfoot{\color{bleu} \thepage / \pageref{LastPage}}

\linespread{1.25}

\begin{document}

\begin{titlepage}
    \begin{center}
    \begin{figure}[!tbp]
      \centering
        \includegraphics[width=0.4\textwidth]{logo.png}
    \end{figure}

    \vspace*{0.5cm}
    \textsc{\LARGE \textbf{Diplôme de l'ENS Paris-Saclay}}\\[1em]    % University Name
    \textsc{\Large Enseignement Supérieur \\ ES 4 -Diffusion des savoir}\\[0.5 cm]   % Course Code
    \vspace{2cm}
    \textsc{\LARGE \textbf{Rapport d'activité}}\\ % University Name
    \vspace{0.2cm}
    \textsc{\LARGE \textbf{Date: \today}}   % University Name
  \end{center}
  \vspace*{3em}
  \hspace*{2cm}
  \begin{minipage}[l]{0.8\linewidth}
    \Large
    \textsc{\large \textbf{Nom:} Comby}\\
    \textsc{\large \textbf{Prénom:} Pierre-Antoine}\\
    \textsc{\large \textbf{Département:} EEA}\\
    \textsc{\large \textbf{E-Mail:} pierre-antoine.comby@ens-paris-saclay.fr}\\
  \end{minipage}

\vspace{-15em}
  \begin{picture}(50,50)
\put(-20,-325){\hbox{\includegraphics[scale=0.6]{footer_left.png}}}
\end{picture}

\begin{picture}(50,50)
\put(400,-270){\hbox{\includegraphics[scale=0.35]{footer_right.png}}}
\end{picture}

\end{titlepage}

\section{Introduction}

Le vendredi 10 janvier 2019, j'ai eu l'opportunité d'intervenir le temps d'une séance de deux heures, devant un groupe de 35 élèves de 1ère Générale en spécialité physique, pour leur donner un cours d'introduction au traitement du signal et ses applications.
Cette intervention fut à l'initiative de Madame Virginie Grosjean-Squinabol, professeure agrégée de sciences physiques au Lycée Jean-Zay à Orléans.

Cette intervention est valorisée par le présent rapport dans le cadre de l'obtention du diplôme de l'ENS Paris-Saclay, pour la validation de l'item ES4 - Diffusion des savoirs.

\section{Préparation de l'activité}

Pour préparer mon intervention devant les élèves lycéens, j'ai pu échanger plusieurs fois avec le professeur de la classe, bénéficier de ses conseils et de ses retours sur mes propositions.
Mme Grosjean-Squinabol m'a ainsi présenté ses attentes et j'ai pu m'assurer que mon travail s'inscrivait bien dans une démarche au long cours ; d'abord pour la préparation du baccalauréat, mais aussi comme un avant-goût des contenus abordés en études supérieures.

\subsection{Objectifs pédagogiques}\label{objectifs-puxe9dagogiques}

Cette intervention avait ainsi plusieurs objectifs :

\begin{itemize}
\item
  introduire de \textbf{nouvelles notions} dans le programme de sciences
  physiques, à l'interface entre physique et mathématiques,
\item
  donner un aperçu des applications du \textbf{traitement du signal},
  développé plus amplement dans les études supérieures, mais déjà
  utilisé au quotidien.
\item
  présenter mon \textbf{parcours post-bac,} ainsi que mon travail de
  recherche actuel.
\end{itemize}

\subsection{Problématiques rencontrées}

Compte tenu du public visé ainsi que du temps qui m'était imparti, il a été nécessaire de s'adapter, pour produire un cours simple, abordable avec des connaissances d'élève de première avec un profil scientifique. Se mettre à la portée des élèves est un travail exigeant :

j'ai remarqué ce fait dès la préparation de mon intervention: impossible de parler de matrice, d'intégrale, de convolution, de nombres complexes...
Ceux-ci sont pourtant des outils omniprésents en traitement du signal.
J'ai donc fait le choix de mettre de côté les nombreuses hypothèses mathématiques et de plutôt privilégier une approche plus empirique, ``avec les mains'', en ajoutant une expérience à mon exposé,pour faciliter et aérer la compréhension de la théorie présentée.

Les difficultés pour expliquer ces notions se sont également retrouvées dans la présentation de mon travail de recherche, que j'avais ainsi choisi d'aborder uniquement à l'oral, lors d'un temps d'échange avec les élèves.

En plus de cette première barrière de connaissances, j'ai également dû contraindre mon cours dans le temps. La formation que j'ai reçu dans ce domaine dépasse largement le millier d'heures.

Il me fallait cibler l'essentiel en 2 heures : je me suis donc restreint à deux notions centrales : la dualité temps /fréquence et la conversion analogique numérique.

\section{Retour sur le déroulement de l'activité}

La séance s'est déroulée en deux parties, d'abord une partie cours, puis un temps d'échange sur le sujet abordé, ainsi que mes travaux de recherche et les études supérieures en sciences physiques.

\subsection{Cours sur le traitement du signal}

Cette partie a largement été préparée en amont de la séance, comme décrit précédemment, pour qu'il y ait une montée graduelle en niveau, partant d'un niveau pré-bac pour finir sur des notions abordées généralement à niveau bac+2.
Cette partie, largement préparée en amont a duré 1h30 -- et non 1h comme initialement prévu -- cela est dû en majeure partie aux questions et interactions la presentation a en effet été particulièrement riche en échanges, notamment sur la fin.
De plus, face à des concepts nouveaux, il m'a été nécessaire de m'y reprendre à plusieurs reprises pour expliquer certain concepts, en les recoupant avec des exemples déjà abordés.

Si le début de la séance s'est déroulé facee à une classe très calme, quelque peu endormie -- ou mal réveillée à 8h du matin- , la sollicitation des élèves pour donner des exemples, ainsi que la démonstration en pratique des théorèmes abordés a permis d'obtenir une classe plus attentive, certains allant même jusqu'à prendre des notes, ce qui n'était pas le cas au début. Je pense que l'effet de nouveauté, ainsi que le changement de paradigme du professeur -- je suis un
étudiant à peine plus vieux qu'eux- , utilisant des références de la pop-culture a notamment joué en ma faveur.

Le professeur et moi avons pu apprécier la grande qualité d'écoute de ce public néophyte.

\subsection{Temps d'échange}

Ce temps d'échange sous formes de question/réponses a abordé des thèmes variés dans le temps qu'il m'était encore imparti (environ trente minutes), d'abord très proches du contenu abordé au préalable, puis sur mon travail de recherche et enfin sur les études supérieures, (niveau
attendu, études à l'étranger,\ldots{}); il est à noter que le même jour se déroulait le forum des étudiants pour tous les lycéens d'Orléans, et que la plupart des élèves s'y rendaient après la séance. Ainsi ce temps d'échange leur a également permis d'avoir déjà un nouveau regard sur les
études supérieures.

\section{Apports personnels}

J'ai appris à expliquer des concepts qui m'intéressent et j'ai découvert comment me mettre à la portée d'un autre public; je m'adresse habituellement à des professeurs ou des camarades de promotion, c'est-à-dire avec des connaissances comparables ou supérieures aux
miennes.

Par ailleurs, bien qu'ayant déjà manipulé LaTeX, je m'en suis cette fois-ci servi pour présenter un cours, ce qui m'a permis de l'appréhender autrement et de développer ma capacité à l'utiliser.

\section{Conclusion}\label{conclusion}

Je retiendrai de cette expérience formatrice de nombreuses compétences pédagogiques, la satisfaction d'une présentation bien rédigée bien que toujours perfectible, et le plaisir de transmettre des connaissances qui me passionnent, dans un domaine pour lequel j'envisage toujours de
poursuivre mon travail de recherche.\\
\emph{
Ci-joints:
\begin{itemize}
\item Attestation d'intervention
\item Support de présentation et notes associées.
\end{itemize}}

\end{document}
%%% Local Variables:
%%% mode: latex
%%% TeX-master: t
%%% ispell-local-dictionary: "francais"
%%% End:
