\documentclass{beamer}
\usepackage[T1]{fontenc}
\usepackage[utf8]{inputenc}
\usepackage{lmodern}
\usepackage[francais]{babel}
\usepackage{graphicx}
\usepackage{ulem}
\usepackage{pgfpages}
\usepackage{float}
\usepackage{subcaption}
\usepackage{pgfplots}
\graphicspath{{./fig/}}
\usetheme[]{ens}
\setbeamertemplate{navigation symbols}{}

%\usetheme[sectiontitle]{crans}
\title{Le traitement du signal \\et le post-bac}
% A subtitle is optional and this may be deleted
\subtitle{}
\author[]{\textsc{Pierre-Antoine Comby}}
\institute{pierre-antoine.comby@ens-paris-saclay.fr}
% \setbeameroption{show notes on second screen}
\date{\today}
\tikzset{every picture/.style={execute at begin picture={\shorthandoff{:;!?};}}}

\pgfplotsset{
  timeplot/.style={
    xlabel={\tiny$t[s]$},ylabel={\tiny$x(t)$},
    axis lines=middle,
    xlabel style={at=(current axis.right of origin), anchor=west},
    ylabel style={at=(current axis.above origin), anchor=south},
    ticks=none,
    enlarge y limits={rel=0.13},
    enlarge x limits={rel=0.07},
  },
  freqplot/.style={
    xlabel={\tiny$f[Hz]$},ylabel={\tiny$A(f)$},
    axis lines=middle,
    xlabel style={at=(current axis.right of origin), anchor=west},
    ylabel style={at=(current axis.above origin), anchor=south},
    enlarge x limits={rel=0.07},
    xtick=\empty,ytick=\empty
  },
  echplot/.style={
    axis lines = middle,
    width=\textwidth,
    height=0.4\textheight,
    xmin = -20 ,xmax = 20, ymin = -0.1, ymax = 1.5,
    x tick label style={font=\footnotesize},
    ytick=\empty,
    xlabel = {\tiny $f$},
  }
}


\begin{document}
\begin{frame}
   \titlepage
\note{Se présenter, énoncé le plan}
 \end{frame}


\begin{frame}{Objectif de la séance}
  \begin{itemize}[<+(1)->]
    \item Donner un avant goût des études supérieures en maths/physique.
    \note<.->[item]{ Exemple de la ``voie royale''/parcours d'excellence. Ne pas négliger les parcours universitaires, plus spécifiques.}
    \item Introduire des notions niveau bac/bac+1 de traitement du signal (utile pour un TPE par ex.)
    \note<.->[item]{Avec le moins de prérequis possible, pas facile, il va falloir rester bien concentré. Ne pas hésiter à lever la main. Donner votre Prénom}
    \item Présenter mon parcours et mon travail actuel.
    \note<.->[item]{Le travail de recherche en bonus}
  \end{itemize}
\end{frame}

\begin{frame}{Curriculum Vitae}
  \begin{description}[<+(1)->]
    \item[2012-2015:] Lycée Charles Péguy, BAC S , ABIBAC.
    \item[2015-2017:] Classe Prépa (Lycée Saint Louis, Paris)
      \begin{itemize}[<.->]
        \item PCSI
        \item PSI*
      \end{itemize}
    \item[2017-2021:] Ecole Normale Supérieure \sout{Cachan} Paris-Saclay
      \begin{itemize}[<.->]
        \item L3 Sciences de l'ingénieur/Physique
        \item M1 Electronique, Electrotechnique, Automatique
        \item \textit{M2 Stage de recherche à Karlsruhe: Tomographie à ultrason, imagerie médicale}
        \item \textcolor{gray}{M2R: Master Traitement du Signal/IA}
      \end{itemize}
    \item[2021-futur:] \textcolor{gray}{Probablemement une thèse}.
  \end{description}
\end{frame}
\note{Un peu de chance, un peu de talent, beaucoup de travail}



\section{Définition et exemples}
\label{sec:label}
\begin{frame}{Le traitement du signal}
  \begin{itemize}
    \item C'est quoi ?
    \item À quoi ça sert?
    \item Comment ça marche?
  \end{itemize}\pause
  \note{On va regarder Les fondements du TS}
  \begin{block}{Définition}
    \begin{itemize}
      \item Un \textbf{signal} est une fonction d'une ou plusieurs variables qui véhicule de l'information sur un phénomène physique.\pause
      \item Un \textbf{système} (de traitement du signal) transforme un ou des signaux en d'autres signaux ou paramètres, dans le but d'en extraire de l'information.
      \end{itemize}
  \end{block}
\end{frame}
\begin{frame}{Des exemples}
  \centering
  \includegraphics[width=\textwidth]{buzz.png}
\end{frame}
\note[enumerate]{
  \item Smart City: 5G, transmission, compression, analyse,
  \item IRM de cerveau -> imagerie médicale: filtrage, segmentation, reconstruction/synthèse\ldots
  \item Vision Artificielle: segmentation, reconnaissance, fiabilité
  \item LHC: les tres grands instruments pour voir le tout petit
  \item Voiture Autonome: protocoles de transmission, sécurité
  \item SpaceX -> aérospatial: communiquer sur des miliers de km.
}
\begin{frame}{Différent type de TS}
  \setlength{\leftmargini}{2.5cm}         %new code: set margin length to 0
  \begin{itemize}
    \item[Mesure:\pause] Estimation de grandeurs caractéristiques\pause
    \note<.(-1)->[item]{grandeurs statistiques}
    \item[Détection:\pause] Extraction d'un signal utile du bruit\pause
    \note<.(-1)->[item]{temps d'arrivée d'un signal}
    \item[Filtrage:\pause] Élimination de composante indésirable\pause
    \note<.(-1)->[item]{Oreilles, séparer basse et aigus, retirer le bruit, Instagram}
    \item[Codage:\pause] Mise en forme pour transmission\pause
    \note<.(-1)->[item]{Toutes les communications numériques}
    \item[Compression:\pause] Réduction de la redondance d'information.\pause
    \note<.(-1)->[item]{Netflix, JPEG(3 à 100 fois plus petit), MP3, faire une fiche de cours}
    \item[Modulation:\pause] Adaptation d'un signal aux caractéristiques d'une voie de transmission\pause
    \note<.(-1)->[item]{radio AM (WW2), FM }
    \item[Segmentation:\pause] Découpage en parties homogènes\pause
    \note<.(-1)->[item]{Imagerie médicale, satellite, météo}
    \item[Reconnaissance:\pause] Identification de formes appartenant à un dictionnaire.\pause
    \note<.(-1)->[item]{Filtre Snapchat}
  \end{itemize}
\end{frame}

\begin{frame}{Différent type de signaux}

  De l'ordre, il faut de l'ordre:

  \begin{itemize}[<+(1)->]
    \item Nombre de variable (1-D (son), 2-D (image), N-D,\ldots)
    \item réel/complexe
    \item périodiques (ex: cos, sin)
    \item Déterministe/aléatoire
    \begin{itemize}[<.->]
      \item déterministe: évolution complètement prédictible
      \item aléatoire: non prédictible, non reproductible.
    \end{itemize}
    \item Continus / Discret
    \begin{itemize}[<.->]
      \item Analogique/Numérique
    \end{itemize}
  \end{itemize}
\end{frame}
\note{On étudie ici seulement des signaux réel, périodique,
  on va étudier/rappeler
  \begin{itemize}
    \item la dualité fréquentielle de tout signal
    \item les bases de la conversion analogique numérique (tous les instruments de mesure physiques)
  \end{itemize}
}
\section{Dualité Fréquentielle}
\begin{frame}[fragile]{Analyse Spectrale}
 \note{La3: 440 Hz,  apporter ordi/téléphone + Enceinte
    \begin{itemize}
      \item Sinus
      \item triangle
      \item Créneau /carré
    \end{itemize}
  }
  \vspace*{-1em}
  \begin{block}{Expérience}
    \begin{itemize}
      \item<1-> 3 sons à la même fréquence, mais différentes formes d'onde.
      \item<6-> On entend d'autres fréquences, sans les voir: \visible<7->{\it un spectre}
    \end{itemize}

  \end{block}\pause[2]
  %% Time
  \begin{columns}
    \begin{column}{0.3\textwidth}
      \centering
      \begin{tikzpicture}
        \begin{axis}
          [width=1.33\textwidth, height=0.4\textheight,
          domain =0:10,
          timeplot]
          \addplot+[no marks, smooth]{sin(360*x/5)};
        \end{axis}
      \end{tikzpicture}\pause[3]
    \end{column}
    \begin{column}{0.3\textwidth}
      \centering
      \begin{tikzpicture}
        \begin{axis}
          [timeplot, width=1.33\textwidth, height=0.4\textheight,]
          \addplot+[mark=none] coordinates {
            (0,0) (1.25,1) (3.75,-1) (6.25,1) (8.75,-1) (10,0)
          };
        \end{axis}
      \end{tikzpicture}\pause[4]
    \end{column}
    \begin{column}{0.3\textwidth}
      \centering
      \begin{tikzpicture}
        \begin{axis}
          [timeplot,
          width=1.33\textwidth, height=0.4\textheight,]
          \addplot+[mark=none,const plot] coordinates {
            (0,1) (2.5,-1) (5,1) (7.5,-1) (10,1)
          };
        \end{axis}
      \end{tikzpicture}\pause[5]
    \end{column}
  \end{columns}
  %% Freq

  \pause[7]
  \begin{columns}
    \begin{column}{.3\textwidth}
       \begin{tikzpicture}
        \begin{axis}
          [freqplot,
          width=1.33\textwidth, height=0.4\textheight, ticks=none,
          xmin=0,xmax=10,ymin=0,ymax=1.2,
          ]
          \draw[blue]
          (axis cs:1,0) -- (axis cs:1,1)   node{$\cdot$};
        \end{axis}
      \end{tikzpicture}
    \end{column}
    \begin{column}{.3\textwidth}
       \begin{tikzpicture}
        \begin{axis}
          [freqplot,width=1.33\textwidth, height=0.4\textheight,
          xmin=0,xmax=10,ymin=0,ymax=1.2,
          ]
          \draw[blue]
          (axis cs:1,0) -- (axis cs:1,1)    node{$\cdot$}
          (axis cs:3,0) -- (axis cs:3,0.11) node{$\cdot$}
          (axis cs:5,0) -- (axis cs:5,0.04) node{$\cdot$}
          (axis cs:7,0) -- (axis cs:7,0.02) node{$\cdot$}
          (axis cs:9,0) -- (axis cs:9,0.012)node{$\cdot$}
          ;
        \end{axis}
      \end{tikzpicture}
    \end{column}
    \begin{column}{.3\textwidth}
       \begin{tikzpicture}
        \begin{axis}
          [freqplot, width=1.33\textwidth, height=0.4\textheight,
          xmin=0,xmax=10,ymin=0,ymax=1.2,
          ]
          \draw[blue]
          (axis cs:1,0) -- (axis cs:1,1)    node{$\cdot$}
          (axis cs:3,0) -- (axis cs:3,0.33) node{$\cdot$}
          (axis cs:5,0) -- (axis cs:5,0.2)  node{$\cdot$}
          (axis cs:7,0) -- (axis cs:7,0.143)node{$\cdot$}
          (axis cs:9,0) -- (axis cs:9,0.111)node{$\cdot$}
          ;
      \end{axis}
    \end{tikzpicture}
  \end{column}
\end{columns}
\note{Les autres sons proviennent d'autres fréquences, }
\end{frame}

\begin{frame}{Décomposition de Fourier}
\note{%
  Joseph Fourier, 1806, préfet de l'Isère. Travail sur l'équation de la chaleur (anneau en métal chauffé)
}%
  \framesubtitle{Les maths}
  \begin{alertblock}{Théorème (Lycée)}
    Toute fonction périodique peut s'écrire comme une somme pondérée de fonction sinusoïdale
    \pause
    \[
      x_{per}(t) = \sum_{n=0}^{+\infty}a_{n}cos(2\pi n f_{0}t)+b_{n}sin(2\pi n f_{0}t)
    \]
  \end{alertblock}\pause
  \begin{itemize}[<+->]
    \item $(a_{n})$ et $(b_{n})$ sont des suites, appelées coefficients de Fourier.
    \item Les coefficients s'obtiennent par produit scalaire de fonctions.
      \note[item]{$a_{n} = \int_{-T/2}^{T/2}x(t)\cos(2\pi nft)dt$}
    \item Théorème généralisable à toutes fonctions à énergie finie ($\sim$ toutes les fonctions en physique).
      \note[item]{$X(f)=\int_{-\infty}^{+\infty}x(t)e^{-2\pi i ft}d t$}
  \end{itemize}
\end{frame}
\begin{frame}[fragile]{Décomposition de Fourier}
  \framesubtitle{Exemple: le créneau}
  \begin{tikzpicture}
    \begin{axis}[timeplot,domain=0:10,
      ymax=1.2,ymin=-1.2,
      height=0.6\textheight,width=1.1\textwidth]
      \visible<1->{
      \addplot[mark=none,const plot] coordinates {
        (0,1) (2.5,-1) (5,1) (7.5,-1) (10,1)
      };}
      \foreach \i [count=\ii] in {1,3,5,15,50}{
        \only<+(1)>{
          \addplot+[mark=none,blue,thick] table [x=x, y=\i, col sep=comma] {creneau.csv};}
      }
    \end{axis}
  \end{tikzpicture}
      \[\hat{x}(t)=
      \visible<2->{\frac{4A}{\pi}\times\left[\sin(2\pi f_{0}t)\right.}
      \visible<3->{+\frac{1}{3}\sin(2\pi 3f_{0}t)}
      \visible<4->{+\frac{1}{5}\sin(2\pi 5f_{0}t)}
      \visible<5->{+\cdots}
      \visible<2->{\left.\right]}\]
\begin{tikzpicture}
\begin{axis}
          [freqplot, width=\textwidth, height=0.3\textheight,
          xmin=0,xmax=15,ymin=0,ymax=1.2,
          xtick={1,3,...,15},         xticklabels={$f_{0}$,$3f_{0}$,$5f_{0}$,$7f_{0}$,$9f_{0}$,$11f_{0}$,$13f_{0}$,$15f_{0}$},
          x tick label style={font=\tiny}
          ]
          \draw<2->[blue,very thick](axis cs:1,0) -- (axis cs:1,1)    node{$\cdot$};
          \draw<3->[blue,very thick](axis cs:3,0) -- (axis cs:3,0.33) node{$\cdot$};
          \draw<4->[blue,very thick](axis cs:5,0) -- (axis cs:5,0.2)  node{$\cdot$};
          \draw<5->[blue,very thick]
          \foreach \x in {7,9,...,15}{
            (axis cs:\x,0) -- (axis cs:\x,1/\x)  node{$\cdot$}
          };
      \end{axis}
  \end{tikzpicture}

\end{frame}
\note{phénomène de Gibbs, du a la discontinuité de la fonction créneau}
\begin{frame}[fragile]{Décomposition de Fourier}
  \framesubtitle{Exemple: triangle}
  \begin{tikzpicture}
    \begin{axis}[timeplot,domain=0:10,
      ymax=1.2,ymin=-1.2,
      height=0.6\textheight,width=1.1\textwidth]
      \visible<1->{
      \addplot[mark=none,] coordinates {
        (0,0) (1.25,1) (3.75,-1) (6.25,1) (8.75,-1) (10,0)
      };}
      \foreach \i [count=\ii] in {1,3,5,15,50}{
        \only<+(1)>{
          \addplot+[mark=none,blue,thick] table [x=x, y=\i, col sep=comma] {triangle.csv};}
      }
    \end{axis}
  \end{tikzpicture}
      \[\hat{x}(t)=\visible<2->{\frac{8A}{\pi^{2}}\times\left[\sin(2\pi f_{0}t)\right.}
      \visible<3->{-\frac{1}{9}\sin(2\pi 3f_{0}t)}
      \visible<4->{+\frac{1}{25}\sin(2\pi 5f_{0}t)}
      \visible<5->{+\cdots}
      \visible<2->{\left.\right]}\]
  \begin{tikzpicture}
\begin{axis}
          [freqplot, width=1.1\textwidth, height=0.3\textheight,
          xmin=0,xmax=15,ymin=0,ymax=1.2,
          xtick={1,3,...,15},         xticklabels={$f_{0}$,$3f_{0}$,$5f_{0}$,$7f_{0}$,$9f_{0}$,$11f_{0}$,$13f_{0}$,$15f_{0}$},
          x tick label style={font=\tiny}
          ]
          \draw<2->[blue,very thick](axis cs:1,0) -- (axis cs:1,1)    node{$\cdot$};
          \draw<3->[blue,very thick](axis cs:3,0) -- (axis cs:3,0.11) node{$\cdot$};
          \draw<4->[blue,very thick](axis cs:5,0) -- (axis cs:5,0.008)  node{$\cdot$};
          \draw<5->[blue,very thick]
          \foreach \x in {7,9,...,15}{
            (axis cs:\x,0) -- (axis cs:\x,1/\x^2)  node{$\cdot$}
          };
      \end{axis}
  \end{tikzpicture}
\end{frame}



\begin{frame}{Dualité temps-fréquence}
  \begin{alertblock}{À retenir}
    \begin{itemize}[<+->]
      \item Tout signal ``physique'' possède un unique signal équivalent dans le domaine \emph{fréquentiel}.
      \item L'opération pour passer de l'un à l'autre est appelée \emph{la transformée de Fourier}
        $X(f)=\int_{-\infty}^{+\infty}x(t)e^{-2\pi i ft}d t$
    \end{itemize}
  \end{alertblock}
  \begin{itemize}[<+->]
    \item Preuve: Bac+3, rigoureux en Bac+5
    \item \emph{Tout} type de signaux (son,image, volume 3D)
    \note[item]{la ou il y a du signal, il y a une TF}
  \end{itemize}\pause
  \begin{block}{Remarque}
    Tout signal périodique possède un spectre discret.
  \end{block}\pause
  La transformée de Fourier est utilisée notamment en:
  \begin{columns}
    \begin{column}{0.5\textwidth}
      \begin{itemize}[<+->]
        \item physique quantique
        \item compression d'image
      \end{itemize}
    \end{column}
    \begin{column}{0.5\textwidth}
      \begin{itemize}[<+->]
        \item cryptographie
        \item ...
      \end{itemize}
    \end{column}
  \end{columns}

\end{frame}

\section{Conversion Analogique Numérique}
\begin{frame}{Analogique vs Numérique}
  \framesubtitle{Exemples}
      \note{Demander 3 exemples Analogique, Numériques}
  \begin{columns}
    \begin{column}{.5\textwidth}
      Signaux Analogiques:\pause[2]
      \begin{itemize}
        \item La voix
        \item Les sillons d'un disque vinyl
        \item La lumière du soleil
        \item La batterie d'un smartphone
      \end{itemize}
      \pause[3]
    \end{column}
    \begin{column}{.5\textwidth}
      Signaux Numériques:\pause[4]
      \begin{itemize}
        \item Les sillons d'un CD
        \item Une image
        \item L'ADN
        \item 100111101
      \end{itemize}
    \end{column}
  \end{columns}
\pause[5]
  \begin{columns}
    \begin{column}{.5\textwidth}
       \begin{block}{}
        \begin{itemize}
          \item Continus
          \item Valeurs quelconques
        \end{itemize}
      \end{block}
    \end{column}
    \begin{column}{.5\textwidth}
      \begin{block}{}
        \begin{itemize}
          \item temps/index discret
          \item valeurs limitées
        \end{itemize}
      \end{block}
    \end{column}
  \end{columns}
\end{frame}

\begin{frame}{Conversion Analogique Numérique}
  \framesubtitle{Généralité}
\includegraphics[width=\textwidth]{fig/discret-continu}
\end{frame}
\note{
  \begin{itemize}
    \item On ne va etudier que l'échantillonnage en détail
    \item La quantification -> faire des arrondis -> pertes d'information. (entropie de shannon hors niveau)
  \end{itemize}

}
\newcommand\xticklabel{}
\newcommand\xtickpos{\empty}
\begin{frame}[fragile]{Echantillonage}
   \begin{block}{Échantillonneur}
      \xticklabel
    \begin{columns}
      \begin{column}{0.55\textwidth}
        \begin{itemize}
          \item Multiplication par une fonction peigne
          \item Période d'échantillonage $T_{e}$
          \note[item]{Période -> Fréquence -> fourier}
        \end{itemize}
      \end{column}
      \visible<2->{
      \begin{column}{0.45\textwidth}
        \centering
        \begin{tikzpicture}
          \node[draw,right=2em](E) {%
            \begin{tikzpicture}[scale=0.5]
              \draw (0,0) -- (2,0);
              \draw[-latex] (0,0) -- (0,1);
              \draw[-latex] (1,0) -- (1,1);
              \draw[-latex] (2,0) -- (2,1);
            \end{tikzpicture}
          };
          \node[left=3em] (In) at (E) {$x_{c}(t)$};
          \node[right=3em] (Out) at (E){$x_{e}(t)$};
          \draw[->] (In) -- (E);
          \draw[->] (E) -- (Out);
        \end{tikzpicture}
      \end{column}}
    \end{columns}
  \end{block}
  { \centering\vspace{2em}\hspace{2em}
  \begin{tikzpicture}
    \only<1-2>{
    \pgfplotsset{xtick=\empty}}
    \only<3->{
    \pgfplotsset{
      xtick={1.5,3,4.5,6,7.5,9},
      xticklabels={$T_e$,$2T_e$,$3T_e$,$4T_e$,$5T_e$,$6T_e$}
    }}
       \begin{axis}
        [axis y line=middle,axis x line=bottom,
        width=\textwidth,height=0.4\textheight, xmin=0,xmax = 10,ymin=-2,ymax=3, ytick =\empty, ylabel={\tiny{\color{blue}$x_c(t)$}\only<3>{,{\color{red}$x_E(t)$}}},
%        xtick = {1.5,3,4.5,6,7.5,9},
 %       xticklabels={$T_e$,$2T_e$,$3T_e$,$4T_e$,$5T_e$,$6T_e$},
    %    xtick={\xtickpos},
     %   \expandafter xticklabels=\expandafter\xticklabel,
        x tick label style={font=\tiny}]

        \addplot+[smooth,no marks] plot coordinates{(0,2) (2,1.5) (3,2) (4,1) (5,-1.5)(6,-1.5)(7,0)(9,1)};
        \draw<3->[dashed,red]
        (axis cs:1.5,-2) -- (axis cs:1.5,1.6)node{$\bullet$}
        (axis cs:3,-2)   -- (axis cs:3,2)node{$\bullet$}
        (axis cs:4.5,-2) -- (axis cs:4.5,-.25)node{$\bullet$}
        (axis cs:6,-2)   -- (axis cs:6,-1.5)node{$\bullet$}
        (axis cs:7.5,-2) -- (axis cs:7.5,.3)node{$\bullet$}
        (axis cs:9,-2)   -- (axis cs:9,1)node{$\bullet$};
      \end{axis}
    \end{tikzpicture}
  }
  \begin{block}<4->{Remarque}
    La transformée de Fourier d'un peigne est un peigne.
  \end{block}
\end{frame}

\begin{frame}[fragile]{Échantillonage}
  \framesubtitle{Spectre d'un signal échantilloné}
  \begin{figure}[H]
  \centering
    \begin{tikzpicture}
    \begin{axis}%
      [echplot,
      ylabel = {\tiny $|X_{c}(f)|$},
      xtick = {-5.5,5.5},
      xticklabels = {$-F_M$, $+F_M$},
      ]
      \addplot+[red,no marks] plot coordinates {(-5.5,0) (-3,1) (-0.5,0) (0.5,0) (3,1) (5.5,0)};
    \end{axis}
  \end{tikzpicture}
  \caption{Spectre du signal continu}
  \note{Fréquence négative : Ca vient de la formulation générale de la transformée de Fourier, symétrique de la partie positive (signaux réel)}
\end{figure}
\begin{figure}
  \centering
  \begin{tikzpicture}
    \begin{axis}%
      [echplot,
      ylabel = {\tiny$|X_{e}(f)|$},
      xtick = {-13,-5.5,5.5,13},
      xticklabels = {$-F_{e}$,$-F_M$, $+F_M$,$F_{e}$},
      ]
      \addplot+[no marks, red] plot coordinates {
        (-5.5,0) (-3,1) (-0.5,0) (0.5,0) (3,1) (5.5,0)
      };
      \addplot+[no marks,blue] plot coordinates {
        (-18.5,0) (-16,1) (-13.5,0) (-12.5,0) (-10,1) (-7.5,0)
      };
      \addplot+[no marks,blue] plot coordinates {
        (7.5,0) (10,1) (12.5,0) (13.5,0) (16,1) (18.5,0)
      };
    \end{axis}
  \end{tikzpicture}
  \caption{Spectre du signal échantilloné}
\end{figure}
\end{frame}

\begin{frame}[fragile]{Echantillonage}
  \framesubtitle{choix de la fréquence d'échantillonage}
  \begin{figure}
  \centering
  \begin{tikzpicture}
    \begin{axis}%
      [echplot,
      ylabel = {\tiny$|X_{e}(f)|$},
      xtick = {-13,-5.5,5.5,13},
      xticklabels = {$-F_{e}$,$-F_M$, $+F_M$,$F_{e}$},
      ]
      \addplot+[no marks, red] plot coordinates {
        (-5.5,0) (-3,1) (-0.5,0) (0.5,0) (3,1) (5.5,0)
      };
      \addplot+[no marks,blue] plot coordinates {
        (-18.5,0) (-16,1) (-13.5,0) (-12.5,0) (-10,1) (-7.5,0)
      };
      \addplot+[no marks,blue] plot coordinates {
        (7.5,0) (10,1) (12.5,0) (13.5,0) (16,1) (18.5,0)
      };
    \end{axis}
  \end{tikzpicture}
  \caption{Spectre du signal échantilloné}
\end{figure}\pause
\begin{figure}[H]
    \centering
    \begin{tikzpicture}
      \begin{axis}%
        [echplot,
        ylabel = {\tiny$|X_E(f)|$},
        xtick = {-10,-5.5,5.5,10},
        xticklabels = {$-F_{e}$,$-F_M$, $+F_M$,$F_{e}$},
        ]
        \addplot+[no marks, red] plot coordinates {
          (-5.5,0) (-3,1) (-0.5,0) (0.5,0) (3,1) (5.5,0)
        };
        \addplot[no marks, dashed ,blue] plot coordinates {
          (-15.5,0) (-13,1) (-10.5,0) (-9.5,0) (-7,1) (-4.5,0)};
        \addplot[no marks, dashed ,blue] plot coordinates {
          (4.5,0) (7,1) (9.5,0) (10.5,0) (13,1) (15.5,0)
        };
        \addplot[no marks, dashed ,blue] plot coordinates {
          (14.5,0) (17,1)};
        \addplot[no marks, dashed ,blue] plot coordinates {
          (-14.5,0) (-17,1)};

        \addplot[no marks,black,thick] plot coordinates {
          (-17,1) (-15.5,0.4) (-14.5,0.4)
          (-13,1) (-10.5,0) (-9.5,0) (-7,1)
          (-5.5,0.4) (-4.5,0.4) (-3,1) (-0.5,0) (0.5,0) (3,1)
          (4.5,0.4)(5.5,0.4) (7,1) (9.5,0) (10.5,0) (13,1)
          (14.5,0.4) (15.5,0.4) (17,1)};

        \coordinate (R) at (axis cs: -5,0.25);
      \end{axis}
      % \draw[latex-] (R)  to[bend left] ++(-2,2) node[above]{\textit{Repliement}};

    \end{tikzpicture}
  \caption{Repliement de spectre}
  \end{figure}
\end{frame}
\begin{frame}{Échantillonage}
  \framesubtitle{Théorème de Shannon}

  \begin{alertblock}{Theorème de Shannon}
    Pour échantillonner de manière \emph{fidèle} un signal, il faut que la fréquence d'échantillonage $F_{e}$ vérifie:
    \[
      F_{e}\ge 2 F_{max}
    \]
  \end{alertblock}\pause
  \begin{itemize}
    \item ``Théorème fondamental du traitement du signal''
    \item Les spectre ``fantômes'' ne parasitent pas le spectre original
    \begin{itemize}
      \item Pas de pertes d'information
      \item On peux reconstruire le signal analogique (CNA)
    \end{itemize}
  \end{itemize}
\end{frame}
\begin{frame}{Conclusion}
  \begin{itemize}
    \item Le traitement du signal est partout
    \item forte intrication avec les mathématiques
    \item
  \end{itemize}
\end{frame}



\section{Vos Question}

\begin{frame}
\begin{tikzpicture}[remember picture,overlay]
\fill[bleuens](current page.north west) rectangle (current page.south east);
\node at ([yshift=+.15\textheight]current page.center) (title)
  {\usebeamerfont{title}\textcolor{white}{Merci pour votre attention}};

\node[below=2em]
  at(title) (subtitle)
  {\usebeamerfont{subtitle}\textcolor{white}{Des questions ?}};

\node
  at ([yshift=-70pt]current page.center) (institute)
  {\usebeamerfont{institute}\textcolor{white}{\insertinstitute}};

\node
  at ([yshift=-50pt]current page.center) (author)
  {\usebeamerfont{author}\textcolor{white}{\insertauthor}};

\node[anchor=north west]
  at (current page.north west) (logo)
  {\usebeamercolor[fg]{titlegraphic}
    \includegraphics[scale=0.3]{ENSPS-revert.png}};
\end{tikzpicture}
\end{frame}
\note{Demander les noms, 1/2 1/2 1/2
  \begin{itemize}
  \item La prépa/L'ENS/Le supérieur
  \item Le TS
  \item Bonus: la recherche, au tableau
  \end{itemize}
}
\end{document}

%%% Local Variables:
%%% mode: latex
%%% TeX-master: t
%%% ispell-local-dictionary: "francais"
%%% End:
